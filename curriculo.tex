\documentclass[a4paper,10pt]{article}
\usepackage[margin=0.5in,nofoot]{geometry}
\usepackage{fontawesome5}
\usepackage{hyperref}
\usepackage{titlesec}
\usepackage{xcolor}

\hypersetup{
    colorlinks=true,
    linkcolor=blue,
    filecolor=blue,
    urlcolor=blue,
    citecolor=blue
}

\titleformat{\section}{\large\bfseries}{\thesection}{1em}{}[\titlerule]
\titlespacing*{\section}{0pt}{*1}{*1}

\newcommand{\entry}[4]{
  \noindent\textbf{#1} \hfill #2 \\
  \noindent\textit{#3} \hfill \textit{#4} \\
  \vspace{2pt}
}

\newcommand{\project}[2]{
  \noindent\textbf{#1} \hfill #2 \\
  \vspace{2pt}
}

\begin{document}

\pagenumbering{gobble}

\noindent
\begin{minipage}[t]{0.5\textwidth}
\textbf{\Large João Pedro Tentis}

\vspace{0.2em}

\noindent \faMapMarker ~ Rio de Janeiro, RJ - Jacarepaguá ~ \faUser ~ 25 anos \\
\end{minipage}%
\begin{minipage}[t]{0.5\textwidth}
\raggedleft
{\color{blue}} {\faPhone \space +55 21 992285653} \quad
{\color{blue}} \href{mailto:jps.tentis@gmail.com}{\faEnvelope \space jps.tentis@gmail.com}

\vspace{0.2em}

{\color{blue}} \href{https://joaotentis.dev}{\space joaotentis.dev}
{\color{blue}} \href{https://github.com/jtentis}{\faGithub \space GitHub} \quad
{\color{blue}} \href{https://linkedin.com/in/jtentis}{\faLinkedin \space LinkedIn} \\
\end{minipage}

\vspace{0.2em}

\section*{Experiência}
\vspace{0.6em}
\entry{Di Santinni}{\faCalendar \space 03/2025 -- presente}{Analista de Sistemas}{\faMapMarker \space Remoto}
\vspace{-1.6em}
\begin{itemize}
\setlength\itemsep{-0.3em}
\item Responsável pela manutenção e evolução do sistema interno da empresa, garantindo seu funcionamento eficiente e adequado às demandas dos usuários.
\item Gerenciamento de banco de dados Oracle, com foco em desempenho, desenvolvimento de consultas SQL, extração de relatórios e integração entre sistemas.
\item Desenvolvimento de soluções de middleware utilizando Python  (Flask, FastAPI e WebScraping) e Node.js para processamento de arquivos XML e consumo de APIs externas, convertendo insumos dos arquivos em requisições nas APIs.
\end{itemize}

\entry{Viva Rio}{\faCalendar \space 01/2024 -- 03/2025 (\textbf{$\sim$ 1 ano e 2 meses})}{Assistente de Informação I}{\faMapMarker \space Presencial}
\vspace{-1.6em}
\begin{itemize}
\setlength\itemsep{-0.3em}
\item Desenvolvimento de bot desktop para WhatsApp com Node.js com Nest.js, Electron.js, TypeScript e TailwindCSS.
\item Desenvolvimento, personalização e \textit{deploy} na HostGator de plataforma EAD com Moodle usando PHP e SCSS.
\item Criação de Dashboards dinâmicos com PowerBI, usando Excel VBA e Python.
\item Arquitetura microsserviços e realização de testes unitários.
\end{itemize}

\entry{Di Santinni}{\faCalendar \space 02/2022 -- 01/2024 (\textbf{$\sim$ 1 ano e 11 meses})}{Estagiário TI}{\faMapMarker \space Remoto}
\vspace{-1.6em}
\begin{itemize}
\setlength\itemsep{-0.3em}
\item Automatização de serviços usando Python usando pyautogui.
\item Controle de planilhas usando Excel e Google Sheets.
\end{itemize}

\entry{Global Hitss}{\faCalendar \space 10/2020 -- 02/2022 (\textbf{$\sim$ 1 ano e 3 meses})}{ Operador de Computador}{\faMapMarker \space Remoto}
\vspace{-1.6em}
\begin{itemize}
\setlength\itemsep{-0.3em}
\item Automatização de processos com Python web scraping.
\item Controle de planilhas usando Excel.
% \item Controle de documentação de serviços terceirizados em sistema intranet.
\end{itemize}

% \entry{Globo}{\faCalendar \space 03/2019 -- 05/2020 (\textbf{$\sim$ 1 ano e 2 meses})}{Jovem Aprendiz - Assistente administrativo}{\faMapMarker \space Presencial}
% \vspace{-1.6em}
% \begin{itemize}
% \setlength\itemsep{-0.3em}
% \item Solicitação de pedidos para cidades cenográficas.
% \item Controle de estoque de produções no geral.
% \item Confecção, tradução e apresentação de reuniões em inglês para fornecedores estrangeiros.
% \end{itemize}

% \entry{Globo}{\faCalendar \space 03/2019 -- 05/2020 (\textbf{$\sim$ 1 ano e 2 meses})}{Administrativo}{\faMapMarker \space Presencial}
% \vspace{-1.6em}
% \begin{itemize}
% \setlength\itemsep{-0.3em}
% \item Solicitação de pedidos para cidades cenográficas.
% \item Controle de estoque de produções no geral.
% \item Confecção, tradução e apresentação de reuniões em inglês para fornecedores estrangeiros.
% \end{itemize}

\section*{Projetos}
\vspace{0.6em}
\project{\href{https://github.com/jtentis/MatchMovie-Native}{\space Match Movie} | \normalfont{\textit{React Native, TypeScript, EXPO, Node.js, Nest.js, PostgreSQL}}}{\textbf{Fevereiro 2025}}
\vspace{-1.6em}
\begin{itemize}
\setlength\itemsep{-0.3em}
\item Desenvolvimento de aplicação (projeto de TCC) \textit{tinder-like} para facilitar escolhas de filmes para grupos de usuários.
\item Utilização de APIs externas como: TheMovieDB, OpenCage Geocoding e Ingresso.com; e uso de tecnologias como Context API, Expo Secure Store e React Navigation.
\item Redirecionamento para salas de cinemas próximas baseado na localidade dos usuários.
\item Telas com padrão UI/UX, interface responsiva e cativante, com foco na experiência do usuário.
\end{itemize}

\project{\href{https://github.com/jtentis/HorseCare}{\space HorseCare} | \normalfont{\textit{PHP, JavaScript, HTML, CSS, MySQL}}}{\textbf{Dezembro 2023}}
\vspace{-1.6em}
\begin{itemize}
\setlength\itemsep{-0.3em}
\item Desenvolvimento de aplicação para hospedagem de cavalos.
\item Cadastro de usuários e funcionários, com autenticação de login.
% \item Interface responsiva, desenvolvida com Figma e seguindo os padrões de UI/UX.
\end{itemize}

\project{\href{https://www.linkedin.com/company/obinvestbrasil/}{\space Commodities} | \normalfont{\textit{Python, Streamlit, CSS}}}{\textbf{Junho 2023}}
\vspace{-1.6em}
\begin{itemize}
\setlength\itemsep{-0.3em}
\item Desenvolvimento de aplicação com Streamlit para análise e variância de commodities e ativos, utilizando matplotlib e pandas.
\item Feita em parceria com \href{https://www.linkedin.com/company/obinvestbrasil/}{\space OBInvest} (Olimpíada brasileira de investimentos).
% \item Extração de informações de ativos e commodities utilizando yfinance.
\end{itemize}

\section*{Educação}
\vspace{0.6em}
\entry{CEFET}{\faCalendar \space 02/2021 -- 02/2025 (\textbf{$\sim$ 4 anos})}{Sistemas de Informação - Bacharel}{\faMapMarker \space Presencial}
\vspace{0.6em}

\section*{Skills}
\vspace{0.6em}
\begin{itemize}
\item \textbf{Linguas:} \textbf{Nativa}: Português | \textbf{Avançado}: Inglês (nível C2) | \textbf{Básico}: Espanhol
\item \textbf{Tecnologias:} JavaScript, TypeScript, HTML, CSS, Excel, Python, Node.js, PostgreSQL, Oracle, MariaDB, MySQL, MongoDB
\item \textbf{Ferramentas:} Git, GitHub, VSCode, Docker, HostGator, Vercel, AWS (Lambda, S3, EC2), CloudFlare, Railway, Cpanel, Figma, Trello, Jest, Playwright, SOLID
\item \textbf{Frameworks:} Nest.js, Electron.js, Express, React, React Native, Expo, Swagger, TailwindCSS, SCSS, VBA, Pandas, Streamlit, Flask, FastAPI
\end{itemize}

\end{document}