\documentclass[a4paper,11pt]{article}

\usepackage[T1]{fontenc}
\usepackage[utf8]{inputenc}
\usepackage[a4paper, margin=18mm]{geometry}
\usepackage{enumitem}
\usepackage{titlesec}
\usepackage[hidelinks]{hyperref}
\usepackage{xcolor}

% Colors matching the PDF template
\definecolor{dark}{HTML}{1A1A1A}
\definecolor{muted}{HTML}{4B5563}
\definecolor{light}{HTML}{6B7280}

\hypersetup{
    colorlinks=true,
    linkcolor=blue,
    filecolor=blue,
    urlcolor=blue,
    citecolor=blue
}

% Section formatting: uppercase, bold, with rule below
\titleformat{\section}
  {\normalfont\large\bfseries\color{dark}\MakeUppercase}
  {}
  {0pt}
  {}
  [\vspace{-0.6em}\rule{\linewidth}{0.4pt}]
\titlespacing*{\section}{0pt}{1.2em}{0.4em}

\newcommand{\project}[2]{
  \noindent\textbf{#1} \hfill #2 \\
  \vspace{2pt}
}

% Remove page numbers
\pagestyle{empty}

% Default text color
\color{dark}

\begin{document}

% --- Name ---
\begin{center}
{\LARGE\bfseries João Pedro Tentis}
\end{center}
\begin{center}
{\color{muted}\large Desenvolvedor Full-Stack | Especialista em React, Node.js e Python}
\end{center}
\begin{center}
{\color{muted}\small jps.tentis@gmail.com \enspace | \enspace +55 (21) 99228-5653} \enspace | \enspace \href{https://joaotentis.dev}{joaotentis.dev}
\end{center}
\vspace{0.2em}
\section*{Resumo Profissional}
Desenvolvedor Full-Stack bilíngue com 5+ anos de experiência em desenvolvimento web e automação, especializado em arquiteturas serverless AWS e aplicações React/Node.js. Experiência comprovada em desenvolvimento de APIs REST, integração de microsserviços e modelagem de bancos de dados SQL, tendo entregado soluções que processam milhares de requisições diárias e automatizado processos reduzindo tempo operacional em até 70\%.

\section*{Experiencia Profissional}
\vspace{0.3em}
\noindent\textbf{Analista de Sistemas} \hfill {\color{light}\small 03/2025 - Presente}

\noindent{\color{light}Di Santinni}
\begin{itemize}[leftmargin=1.5em, itemsep=1pt, parsep=0pt, topsep=2pt]
  \item Desenvolvi e mantive APIs REST em Python (Flask, FastAPI) e Node.js processando arquivos XML e integrando com sistemas externos, reduzindo tempo de processamento em 60\%
  \item Otimizei consultas SQL em banco de dados Oracle gerenciando mais de 10 milhões de registros, melhorando performance de relatórios em 45\%
  \item Implementei soluções de middleware para consumo de APIs externas convertendo dados XML em requisições HTTP automatizadas, eliminando processos manuais
  \item Arquitetei integração entre sistemas legados e novas aplicações garantindo zero downtime durante migrações
\end{itemize}

\vspace{0.3em}
\noindent\textbf{Assistente de Informação I} \hfill {\color{light}\small 01/2024 - 03/2025}

\noindent{\color{light}Viva Rio}
\begin{itemize}[leftmargin=1.5em, itemsep=1pt, parsep=0pt, topsep=2pt]
  \item Desenvolvi aplicação desktop para WhatsApp utilizando Node.js, Nest.js, Electron.js, TypeScript e TailwindCSS, atendendo mais de 1.200 usuários mensais
  \item Implementei e realizei deploy de plataforma EAD Moodle personalizada usando PHP e SCSS na HostGator, suportando 500+ alunos simultâneos
  \item Criei dashboards dinâmicos em PowerBI integrados com Python e Excel VBA processando e visualizando conjuntos de dados com 50.000+ linhas
  \item Projetei arquitetura de microsserviços com cobertura de testes unitários superior a 80\%, reduzindo bugs em produção em 55\%
\end{itemize}

\vspace{0.3em}
\noindent\textbf{Estagiário TI} \hfill {\color{light}\small 02/2022 - 01/2024}

\noindent{\color{light}Di Santinni}
\begin{itemize}[leftmargin=1.5em, itemsep=1pt, parsep=0pt, topsep=2pt]
  \item Automatizei 15+ processos operacionais usando Python (pyautogui) reduzindo tempo de execução de tarefas repetitivas em 70\%
  \item Desenvolvi scripts de manipulação e integração de planilhas Excel e Google Sheets processando mais de 10.000 registros diários
  \item Implementei soluções de extração e transformação de dados otimizando workflows internos e aumentando produtividade da equipe em 40\%
\end{itemize}

\vspace{0.3em}
\noindent\textbf{Operador de Computador} \hfill {\color{light}\small 10/2020 - 02/2022}

\noindent{\color{light}Global Hitss}
\begin{itemize}[leftmargin=1.5em, itemsep=1pt, parsep=0pt, topsep=2pt]
  \item Automatizei processos de web scraping com Python coletando e processando dados de múltiplas fontes, reduzindo trabalho manual em 65\%
  \item Gerenciei e organizei bases de dados em Excel com mais de 20.000 linhas garantindo integridade e disponibilidade de informações críticas
  \item Otimizei rotinas operacionais através de scripts automatizados aumentando eficiência do setor em 50\%
\end{itemize}

\section*{Projetos}
\vspace{0.6em}
\project{\href{https://github.com/jtentis/MatchMovie-Native}{\space Match Movie} | \normalfont{\textit{React Native, TypeScript, EXPO, Node.js, Nest.js, PostgreSQL}}}{\textbf{Fevereiro 2025}}
\vspace{-1.6em}
\begin{itemize}
\setlength\itemsep{-0.3em}
\item Desenvolvi uma aplicação (projeto de TCC) \textit{tinder-like} para facilitar escolhas de filmes para grupos de usuários.
\item Utilizei APIs externas como: TheMovieDB, OpenCage Geocoding e Ingresso.com; e uso de tecnologias como Context API, Expo Secure Store e React Navigation.
\item A aplicação redireciona para salas de cinemas próximas baseado na localidade dos usuários.
\item Telas com padrão UI/UX, interface responsiva e cativante, com foco na experiência do usuário.
\end{itemize}

\project{\href{https://github.com/jtentis/HorseCare}{\space HorseCare} | \normalfont{\textit{PHP, JavaScript, HTML, CSS, MySQL}}}{\textbf{Dezembro 2023}}
\vspace{-1.6em}
\begin{itemize}
\setlength\itemsep{-0.3em}
\item Desenvolvi uma aplicação para hospedagem de cavalos.
\item Cadastro de usuários e funcionários, com autenticação de login.
% \item Interface responsiva, desenvolvida com Figma e seguindo os padrões de UI/UX.
\end{itemize}

\project{\href{https://www.linkedin.com/company/obinvestbrasil/}{\space Commodities} | \normalfont{\textit{Python, Streamlit, CSS}}}{\textbf{Junho 2023}}
\vspace{-1.6em}
\begin{itemize}
\setlength\itemsep{-0.3em}
\item Desenvolvi uma aplicação web com Streamlit para análise e variância de commodities e ativos, utilizando matplotlib e pandas.
\item Feita em parceria com \href{https://www.linkedin.com/company/obinvestbrasil/}{\space OBInvest} (Olimpíada brasileira de investimentos).
% \item Extração de informações de ativos e commodities utilizando yfinance.
\end{itemize}

\section*{Habilidades}
\vspace{0.6em}
\begin{itemize}
\item \textbf{Linguas:} \textbf{Nativa}: Português | \textbf{Avançado}: Inglês (nível C2) | \textbf{Básico}: Espanhol
\item \textbf{Linguagens:} Python, JavaScript, TypeScript, SQL (PostgreSQL, Oracle, MySQL, MariaDB), HTML, CSS
\item \textbf{Frameworks \& Libs:} React, Next.js, Nest.js, Node.js, Express, Flask, FastAPI, React Native, Expo, Redux, Axios, Pandas, Streamlit, Electron.js
\item \textbf{Cloud \& Infra:} AWS (Lambda, S3, EC2), Docker, Serverless, CI/CD, Vercel, HostGator, CloudFlare, Railway, Cpanel
\item \textbf{Ferramentas \& Tecnologias:} Git, GitHub, VSCode, Prisma, TailwindCSS, SCSS, Jest, Playwright, Swagger, Figma, Trello, Excel, VBA, PWA, APIs REST
\end{itemize}

\section*{Formacao Academica}
\vspace{0.3em}
\noindent\textbf{Bacharelado em Sistemas de Informação} \hfill {\color{light}\small 2021 - 2025}

\noindent{\color{light}CEFET}

\end{document}
